
% This LaTeX was auto-generated from an M-file by MATLAB.
% To make changes, update the M-file and republish this document.



    
    
      \subsection{getPlotDistr\_cp.m}

\begin{par}
\textbf{Summary:} Compute means and covariances of the Cartesian coordinates of the tips both the inner and outer pendulum assuming that the joint state $x$ of the cart-double-pendulum system is Gaussian, i.e., $x\sim N(m, s)$
\end{par} \vspace{1em}

\begin{verbatim}   function [M, S] = getPlotDistr_cp(m, s, ell)\end{verbatim}
    \begin{par}
\textbf{Input arguments:}
\end{par} \vspace{1em}
\begin{verbatim}m       mean of full state                                    [4 x 1]
s       covariance of full state                              [4 x 4]
ell     length of pendulum\end{verbatim}
\begin{verbatim}Note: this code assumes that the following order of the state:
       1: cart pos.,
       2: cart vel.,
       3: pendulum angular velocity,
       4: pendulum angle\end{verbatim}
\begin{par}
\textbf{Output arguments:}
\end{par} \vspace{1em}
\begin{verbatim}M      mean of tip of pendulum                               [2 x 1]
S      covariance of tip of pendulum                         [2 x 2]\end{verbatim}
\begin{par}
Copyright (C) 2008-2013 by Marc Deisenroth, Andrew McHutchon, Joe Hall, and Carl Edward Rasmussen.
\end{par} \vspace{1em}
\begin{par}
Last modification: 2013-03-27
\end{par} \vspace{1em}


\subsection*{High-Level Steps} 

\begin{enumerate}
\setlength{\itemsep}{-1ex}
   \item Augment input distribution to complex angle representation
   \item Compute means of tips of pendulums (in Cartesian coordinates)
   \item Compute covariances of tips of pendulums (in Cartesian coordinates)
\end{enumerate}

\begin{lstlisting}
function [M, S] = getPlotDistr_cp(m, s, ell)
\end{lstlisting}


\subsection*{Code} 


\begin{lstlisting}
% 1. Augment input distribution to complex angle representation
[m1 s1 c1] = gTrig(m,s,4,ell); % map input distribution through sin/cos
m1 = [m; m1];        % mean of joint
c1 = s*c1;           % cross-covariance between input and prediction
s1 = [s c1; c1' s1]; % covariance of joint

% 2. Compute means of tips of pendulums (in Cartesian coordinates)
M = [m1(1)+m1(5); -m1(6)];

% 3. Compute covariances of tips of pendulums (in Cartesian coordinates)
s11 = s1(1,1) + s1(5,5) + s1(1,5) + s1(5,1); % x+l sin(theta)
s22 = s1(6,6); % -l*cos(theta)
s12 = -(s1(1,6)+s1(5,6)); % cov(x+l*sin(th), -l*cos(th)

S = [s11 s12; s12' s22];
try
  chol(S);
catch
  warning('matrix S not pos.def. (getPlotDistr)');
  S = S + (1e-6 - min(eig(S)))*eye(2);
end
\end{lstlisting}
