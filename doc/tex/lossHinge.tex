
% This LaTeX was auto-generated from an M-file by MATLAB.
% To make changes, update the M-file and republish this document.



    
    
      \subsection{lossHinge.m}

\begin{par}
\textbf{Summary:} Function to compute the moments and derivatives of the loss of a Gaussian distributed point under a double hinge loss function. The loss function has slope -/+a and corners b1 and b2. The function also calculates derivatives of the loss w.r.t. the state distribution.
\end{par} \vspace{1em}
\begin{par}
Graph:          \ensuremath{\backslash}                   /           \ensuremath{\backslash}                 /            \ensuremath{\backslash}               /             \ensuremath{\backslash}\_\_\_\_\_\_\_\_\_\_\_\_\_/             b1           b2
\end{par} \vspace{1em}
\begin{par}
To use a single hinge b1 or b2 can be set to -Inf or +Inf respectively.
\end{par} \vspace{1em}
\begin{par}
Note, this function is only analytic for 1D inputs. To apply this loss function to multiple variables, use the lossAdd function.
\end{par} \vspace{1em}
\begin{verbatim}function [L dLdm dLds S dSdm dSds C dCdm dCds dLdb] = lossHinge(cost, m, s)\end{verbatim}
\begin{par}
\textbf{Input arguments:}
\end{par} \vspace{1em}
\begin{verbatim}cost
  .fcn      @lossHinge - called to get here
  .a        slope of loss function
  .b        corner points of loss function                     [1   x    2 ]
m             input mean                                       [D   x    1 ]
S             input covariance matrix                          [D   x    D ]\end{verbatim}
\begin{par}
\textbf{Output arguments:}
\end{par} \vspace{1em}
\begin{verbatim}L               expected loss                                  [1   x    1 ]
dLdm            derivative of L wrt input mean                 [1   x    D ]
dLds            derivative of L wrt input covariance           [1   x   D^2]
S               variance of loss                               [1   x    1 ]
dSdm            derivative of S wrt input mean                 [1   x    D ]
dSds            derivative of S wrt input covariance           [1   x   D^2]
C               inv(S) times input-output covariance           [D   x    1 ]
dCdm            derivative of C wrt input mean                 [D   x    D ]
dCds            derivative of C wrt input covariance           [D   x   D^2]\end{verbatim}
\begin{par}
Copyright (C) 2008-2013 by Marc Deisenroth, Andrew McHutchon, Joe Hall, and Carl Edward Rasmussen.
\end{par} \vspace{1em}
\begin{par}
Last modified: 2013-03-06
\end{par} \vspace{1em}


\subsection*{High-Level Steps} 

\begin{enumerate}
\setlength{\itemsep}{-1ex}
   \item Expected cost
   \item Variance of cost
   \item inv(s)* cov(x,L)
\end{enumerate}

\begin{lstlisting}
function [L dLdm dLds S dSdm dSds C dCdm dCds dLdb] = lossHinge(cost, m, s)
\end{lstlisting}


\subsection*{Code} 


\begin{lstlisting}
D = length(m);
if D > 1;
    error(['lossHinge only defined for 1D inputs, use lossAdd to '...
                                     'concatenate multiple 1D loss functions']);
end

a = cost.a;
b = cost.b(:)' - m(:)'; I = ~isinf(b); % centralize
eb = exp(-b.^2/2/s); erfb = erf(b/sqrt(2*s));
c = sqrt(s/pi/2);

% 1. Expected Loss
% int_{-inf}^{b1-m} -a*(x-b1+m)*N(0,S)  +  int_{b2-m}^inf a*(x-b2+m)*N(0,S)
L = a*(b/2.*erfb + c*eb + b.*[1,-1]/2);
L = sum(L(I));

if nargout > 1
    % Derivative w.r.t. m
    dLdb = a/2*(erfb + [1,-1]);
    dLdm = sum(dLdb(I)*-1);

    % Derivative w.r.t. S
    dc = 1/(2*sqrt(2*pi*s));
    dLds = a*sum(eb)*dc;
end

% 2. Variance of Loss
if nargout > 3
    S = a^2*((b.^2+s).*(1+[1,-1].*erfb)/2 + [1,-1].*b*c.*eb);
    S = sum(S(I)) - L^2;

    erfbdm = -sqrt(2/pi/s)*eb; erfbds = -b.*eb/sqrt(2*pi*s^3);
    dSdm = a^2*(-b.*(1+[1,-1].*erfb) + (b.^2+s).*[1,-1].*erfbdm/2 + ...
                                                    [1,-1]*c.*eb.*(-1+b.^2/s));
    dSdm = sum(dSdm(I)) - 2*L*dLdm;
    dSds = a^2/2*((1+[1,-1].*erfb) + (b.^2+s).*[1,-1].*erfbds + ...
                                   [2,-2].*b*dc.*eb + [1,-1].*b.^3/s^2*c.*eb);
    dSds = sum(dSds(I)) - 2*L*dLds;
end

% 3. inv(s)* covariance between input and cost
if nargout > 6
   C = a*([-1 1] - erfb)/2;
   C = sum(C);

   dCdm = -a/2*sum(erfbdm);
   dCds = -a/2*sum(erfbds(I));
end
\end{lstlisting}
