\documentclass[11pt, a4paper]{article}
\renewcommand{\rmdefault}{psbx}
\usepackage[utf8]{inputenc}
\usepackage[T1]{fontenc}
\usepackage{textcomp}
\usepackage{eulervm}
\usepackage{amsmath}
\usepackage{amssymb}
\usepackage{wrapfig}

\setlength{\textwidth}{160mm}
\setlength{\oddsidemargin}{0mm}
\setlength{\parindent}{0 mm}

\renewcommand{\vec}{\boldsymbol}
\newcommand{\mat}{\boldsymbol}
\newcommand{\E}{{\mathbb E}}
\newcommand{\V}{{\mathbb V}}

\title{Policy Squashing}
\author{Marc Peter Deisenroth}
\date{\today}


\begin{document}
\maketitle


We consider a third-order Fourier series expansion 
%
\begin{align*}
s(x) =  a\sin(x) + b\sin(3x)
\end{align*}
%
%
% \begin{wrapfigure}{r}{0.5\hsize}
% \vspace{-5mm}
% \centering
% \includegraphics[width = \hsize]{./figures/squashing_function}
% \caption{Different squashing functions.}
% \label{fig:squashing function}
% \end{wrapfigure}
% 
of a trapezoidal wave with modified Fourier coefficients $a$ and $b$
that satisfy the following constraints: First, when restricted to the
interval $[-\tfrac{\pi}{2}, \tfrac{\pi}{2}]$, $s(x)$ is monotonically
increasing, i.e., $s^\prime(x) \geq 0$ with $s^\prime(x)=0$ if and
only if $x\in\{-\tfrac{\pi}{2}, \tfrac{\pi}{2}\}$. Second, at the
boundaries $\pm\tfrac{\pi}{2}$, we require that $s(\tfrac{\pi}{2}) =
1$ and $s(-\tfrac{\pi}{2}) = -1$.  To closely approximate a (normalized)
trapezoidal wave, we additionally require that $s$ has stationary
points at the boundaries $\pm\tfrac{\pi}{2}$, i.e.,
$s^{\prime\prime}(\pm\tfrac{\pi}{2}) = 0$. The first and second
derivatives of $s(x)$ are
%
\begin{align*}
s^\prime(x) &= a\cos (x) + 3b\cos(3x)\,,\\
s^{\prime\prime}(x) &= -a\sin(x) - 9b\sin(3x)\,.
\end{align*}
 %
From $s^\prime(\pm\tfrac{\pi}{2})=0$, we obtain $a = 1+b$. For
$b>\tfrac{1}{8}$, $s(x)$ is no longer monotonically increasing in
$[-\tfrac{\pi}{2}, \tfrac{\pi}{2}]$: The first derivative
$s^\prime(x)$ crosses 0 for $b=-\tfrac{\cos x}{\cos x +
  3\cos(3x)}$. For $b>\tfrac{1}{8}$, $s(x)$ has a local optimum for
$x\in(-\tfrac{\pi}{2},\tfrac{\pi}{2})$. Therefore,
$b\in[0,\tfrac{1}{8}$ are valid choices.  Using now any of the
stationary point constraints yields $s^{\prime\prime}(-\tfrac{\pi}{2})
= 1+b - 9b = 0 \Leftrightarrow b = \tfrac{1}{8}$ and $a = 1+b =
\tfrac{9}{8}$. Hence, our squashing function is
%
\begin{align*}
s(x) = \tfrac{9}{8}\sin(x) + \tfrac{1}{8}\sin(3x)\,.
\end{align*}
%
For $b=0$, the squashing function equals the
sine wave (red), the black dashed curve is the squashing function for
$b=\tfrac{1}{20}$. However, only for $b=\tfrac{1}{8}$, the squashing
function possesses stationary points at the boundaries.


\end{document}

%%% Local Variables: 
%%% mode: latex
%%% TeX-master: t
%%% End: 
